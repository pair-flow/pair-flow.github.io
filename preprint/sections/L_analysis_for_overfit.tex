
\section{Analysis for the Overfitting in Image Domains}
\label{sec:overfit_analysis}

In this section, we evaluate our method using FID computed on the test sets of two image domains: CIFAR-10~\citep{krizhevsky2009learning} and MNIST-Binary~\citep{lecun2002gradient}. For CIFAR-10, we additionally report FID scores measured with DINOv2~\citep{oquab2024dinov2}. The overall results are summarized in~\Tabref{tab:cifar_fid_test} and~\Tabref{tab:mnist_binary_fid_test}. Across all evaluation metrics, the performance trend is consistent with our main findings—\methodname{} delivers improved generation quality over the baseline, with especially strong gains in the few-step generation regime.


\begin{table}[H]
\small
\centering
% color to blue
% \captionsetup{font={color=correctblue}} 
% \color{correctblue}
% \arrayrulecolor{correctblue}
% color to blue
\caption{FID and FID-Dino scores ($\downarrow$) on test dataset for CIFAR-10~\citep{krizhevsky2009learning} comparison across extended NFE steps (1 to 1024). Best values are bolded.}
\label{tab:cifar_fid_test}
\resizebox{\textwidth}{!}{
\begin{tabular}{l|ccccccccccc}
\toprule
NFE & 1 & 2 & 4 & 8 & 16 & 32 & 64 & 128 & 256 & 512 & 1024 \\
\midrule
Method & \multicolumn{11}{c}{FID ($\downarrow$)} \\
\midrule
UDLM & 306.45 & 296.77 & 266.64 & 178.11 & 114.04 & 80.40 & 62.70 & 53.83 & 50.96 & 47.61 & 47.16 \\
\methodname{} & \textbf{235.65} & \textbf{247.18} & \textbf{209.94} & \textbf{137.16} & \textbf{94.24} & \textbf{67.53} & \textbf{51.48} & \textbf{43.79} & \textbf{42.44} & \textbf{40.85} & \textbf{39.92} \\
\midrule
Method & \multicolumn{11}{c}{FID-DINOv2 ($\downarrow$)} \\
\midrule
UDLM & 2448.46 & 2410.65 & 1975.58 & 1344.44 & 959.39 & 755.80 & 646.47 & 598.53 & 598.54 & 553.17 & 560.75 \\
\methodname{} & \textbf{2059.26} & \textbf{1988.21} & \textbf{1626.08} & \textbf{1127.30} & \textbf{828.09} & \textbf{623.89} & \textbf{530.70} & \textbf{486.59} & \textbf{484.27} & \textbf{470.33} & \textbf{470.05} \\
\bottomrule
\end{tabular}
}
\end{table}

% \begin{table}[H]
% \small
% \centering
% \caption{FID scores ($\downarrow$) on the MNIST-Binary~\citep{lecun2002gradient} test set across various NFE steps (1 to 64).}
% \label{tab:mnist_binary_fid_test}
% \resizebox{\textwidth}{!}{
% \begin{tabularx}{\textwidth}{l|YYYYYYY}
% \toprule
% Method & 1 & 2 & 4 & 8 & 16 & 32 & 64 \\
% \midrule
% UDLM & 129.05 & 42.17 & 11.42 & 6.18 & 5.13 & 5.37 & 5.50 \\
% \methodname{} & 42.87 & 17.37 & 9.62 & 6.36 & 5.80 & 5.51 & 5.24 \\
% \midrule
% UDLM+DCD & 57.85 & 19.23 & 9.82 & 8.41 & 7.87 & 7.12 & 7.38 \\
% \methodname{}+DCD & 19.56 & 13.06 & 10.90 & 8.55 & 7.40 & 7.22 & 7.85 \\
% \midrule
% UDLM+ReDi & 19.08 & 10.79 & 8.77 & 7.01 & 6.89 & 6.57 & 6.61 \\
% \methodname{}+ReDi & 13.73 & 9.59 & 8.98 & 7.24 & 7.22 & 6.98 & 7.12 \\
% \bottomrule
% \end{tabularx}
% }
% \end{table}

\begin{table}[H]
\small
\centering
% color to blue
% \captionsetup{font={color=correctblue}} 
% \color{correctblue}
% \arrayrulecolor{correctblue}
% color to blue
\caption{FID scores ($\downarrow$) on the MNIST-Binary~\citep{lecun2002gradient} test set across various NFE steps (1 to 64). Best values are bolded.}
\label{tab:mnist_binary_fid_test}
\resizebox{\textwidth}{!}{
\begin{tabularx}{\textwidth}{l|YYYYYYY}
\toprule
Method & 1 & 2 & 4 & 8 & 16 & 32 & 64 \\
\midrule
UDLM & 129.05 & 42.17 & 11.42 & \textbf{6.18} & \textbf{5.13} & \textbf{5.37} & 5.50 \\
\methodname{} & \textbf{42.87} & \textbf{17.37} & \textbf{9.62} & 6.36 & 5.80 & 5.51 & \textbf{5.24} \\
\midrule
UDLM+DCD & 57.85 & 19.23 & \textbf{9.82} & \textbf{8.41} & 7.87 & \textbf{7.12} & \textbf{7.38} \\
\methodname{}+DCD & \textbf{19.56} & \textbf{13.06} & 10.90 & 8.55 & \textbf{7.40} & 7.22 & 7.85 \\
\midrule
UDLM+ReDi & 19.08 & 10.79 & \textbf{8.77} & \textbf{7.01} & \textbf{6.89} & \textbf{6.57} & \textbf{6.61} \\
\methodname{}+ReDi & \textbf{13.73} & \textbf{9.59} & 8.98 & 7.24 & 7.22 & 6.98 & 7.12 \\
\bottomrule
\end{tabularx}
}
\end{table}


We further assess potential memorization by measuring nearest-neighbor distances with respect to the training set. For MNIST-Binary~\citep{lecun2002gradient}, we compute pixel-wise $\ell_2$ distances, whereas for CIFAR-10~\citep{krizhevsky2009learning}, we evaluate both $\ell_2$ distance and cosine similarity between features extracted using DINOv2~\citep{oquab2024dinov2}. As summarized in~\Tabref{tab:cifar_nn_distance} and~\Tabref{tab:mnist_binary_nn_distance}, across all evaluation settings, the nearest-neighbor distances of \methodname{} are comparable to or slightly larger than those of the baseline. These results support the conclusion that our method does not suffer from severe overfitting or excessive memorization of the training data.


\begin{table}[H]
\small
\centering
% color to blue
% \captionsetup{font={color=correctblue}} 
% \color{correctblue}
% \arrayrulecolor{correctblue}
% color to blue
\caption{Comparison of $\ell_2$ and Dino~\citep{oquab2024dinov2} Cosine nearest neighbor distance on the CIFAR-10~\citep{krizhevsky2009learning} training set across extended NFE steps (1 to 1024). Best values are bolded.}
\label{tab:cifar_nn_distance}
\resizebox{\textwidth}{!}{
\begin{tabular}{l|ccccccccccc}
\toprule
NFE & 1 & 2 & 4 & 8 & 16 & 32 & 64 & 128 & 256 & 512 & 1024 \\
\midrule
Metric & \multicolumn{11}{c}{$\ell_2$ ($\uparrow$)} \\
\midrule
UDLM & 7.97 & \textbf{9.13} & \textbf{10.06} & \textbf{10.06} & \textbf{9.40} & \textbf{8.94} & \textbf{8.63} & 8.41 & 8.29 & 8.29 & 8.22 \\
\methodname{} & \textbf{8.03} & 8.76 & 9.42 & 9.56 & 9.18 & 8.75 & \textbf{8.63} & \textbf{8.55} & \textbf{8.51} & \textbf{8.52} & \textbf{8.52} \\
\midrule
Metric & \multicolumn{11}{c}{Cosine(DINOv2) ($\downarrow$)} \\
\midrule
UDLM & \textbf{0.242} & \textbf{0.227} & \textbf{0.231} & 0.241 & 0.237 & 0.238 & 0.235 & 0.237 & 0.237 & \textbf{0.235} & 0.237 \\
\methodname{} & 0.245 & 0.228 & 0.235 & \textbf{0.239} & \textbf{0.236} & \textbf{0.232} & \textbf{0.232} & \textbf{0.230} & \textbf{0.232} & \textbf{0.235} & \textbf{0.233} \\
\bottomrule
\end{tabular}
}
\end{table}

% \begin{table}[H]
% \small
% \centering
% % color to blue
% \captionsetup{font={color=correctblue}} 
% \color{correctblue}
% \arrayrulecolor{correctblue}
% % color to blue
% \caption{Comparison of $\ell_2$ nearest neighbor distance on the MNIST-Binary~\citep{lecun2002gradient} training set across extended NFE steps (1 to 64). Best values are bolded.}
% \label{tab:mnist_binary_nn_distance}
% \resizebox{\textwidth}{!}{
% \begin{tabularx}{\textwidth}{l|YYYYYYY}
% \toprule
% Method & 1 & 2 & 4 & 8 & 16 & 32 & 64 \\
% \midrule
% UDLM & 8.18 & 7.06 & 6.55 & 6.42 & 6.27 & 6.29 & 6.25 \\
% \methodname{} & 7.36 & 6.95 & 6.63 & 6.42 & 6.26 & 6.34 & 6.30 \\
% \midrule
% UDLM+DCD & 7.24 & 6.78 & 6.57 & 6.50 & 6.31 & 6.40 & 6.39 \\
% \methodname{}+DCD & 7.64 & 7.34 & 7.11 & 6.91 & 6.74 & 6.76 & 6.79 \\
% \midrule
% UDLM+ReDi & 7.14 & 6.81 & 6.49 & 6.24 & 6.10 & 6.14 & 6.08 \\
% \methodname{}+ReDi & 6.84 & 6.57 & 6.35 & 6.09 & 5.95 & 5.97 & 5.96 \\
% \bottomrule
% \end{tabularx}
% }
% \end{table}

\begin{table}[H]
\small
\centering
% color to blue
% \captionsetup{font={color=correctblue}} 
% \color{correctblue}
% \arrayrulecolor{correctblue}
% color to blue
\caption{Comparison of $\ell_2$ nearest neighbor distance on the MNIST-Binary~\citep{lecun2002gradient} training set across extended NFE steps (1 to 64). Best values are bolded.}
\label{tab:mnist_binary_nn_distance}
\resizebox{\textwidth}{!}{
\begin{tabularx}{\textwidth}{l|YYYYYYY}
\toprule
Method & 1 & 2 & 4 & 8 & 16 & 32 & 64 \\
\midrule
UDLM & \textbf{8.18} & \textbf{7.06} & 6.55 & \textbf{6.42} & \textbf{6.27} & 6.29 & 6.25 \\
\methodname{} & 7.36 & 6.95 & \textbf{6.63} & \textbf{6.42} & 6.26 & \textbf{6.34} & \textbf{6.30} \\
\midrule
UDLM+DCD & 7.24 & 6.78 & 6.57 & 6.50 & 6.31 & 6.40 & 6.39 \\
\methodname{}+DCD & \textbf{7.64} & \textbf{7.34} & \textbf{7.11} & \textbf{6.91} & \textbf{6.74} & \textbf{6.76} & \textbf{6.79} \\
\midrule
UDLM+ReDi & \textbf{7.14} & \textbf{6.81} & \textbf{6.49} & \textbf{6.24} & \textbf{6.10} & \textbf{6.14} & \textbf{6.08} \\
\methodname{}+ReDi & 6.84 & 6.57 & 6.35 & 6.09 & 5.95 & 5.97 & 5.96 \\
\bottomrule
\end{tabularx}
}
\end{table}
