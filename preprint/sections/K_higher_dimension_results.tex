
\section{Application for more Complex Systems}
\label{sec:higher_dimension_results}

In this section, we evaluate our method on a higher-dimensional dataset. Specifically, we use the FFHQ~\citep{karras2019styleffhq} dataset, downsampled to $64\times64$. Following the same protocol as in our main experiments, we generate 5{,}000 samples across varying timesteps and report the FID computed against the training set. The results of this experiment are provided in~\Tabref{tab:ffhq_fid}. All training hyperparameters are kept identical to those used in the CIFAR-10 experiments described in Section~\ref{sec:results}.


\begin{table}[H]
\small
\centering
% color to blue
% \captionsetup{font={color=correctblue}} 
% \color{correctblue}
% \arrayrulecolor{correctblue}
% color to blue
\caption{Comparison of FID scores ($\downarrow$) on FFHQ~\citep{karras2019styleffhq} downsampled to $64\times 64$ resolution across extended NFE steps (1 to 1024). Best values per column are highlighted in bold.}
\label{tab:ffhq_fid}
% 데이터 열이 많으므로 텍스트 폭에 맞춰 자동 축소 (graphicx 패키지 필요)
\resizebox{\textwidth}{!}{
\begin{tabular}{l|ccccccccccc}
\toprule
Method & 1 & 2 & 4 & 8 & 16 & 32 & 64 & 128 & 256 & 512 & 1024 \\
\midrule
UDLM & 403.04 & 399.26 & 363.97 & 273.31 & 153.71 & 97.87 & 74.85 & 63.93 & 59.28 & 55.99 & 55.30 \\
\methodname{} & \textbf{394.14} & \textbf{368.36} & \textbf{329.13} & \textbf{243.88} & \textbf{140.05} & \textbf{90.85} & \textbf{69.67} & \textbf{59.86} & \textbf{56.52} & \textbf{54.19} & \textbf{53.18} \\
\bottomrule
\end{tabular}
}
\end{table}

We adopt the LM1B~\citep{chelba2013lm1b} dataset to evaluate our method under a substantially larger vocabulary size and training corpus. The text corpus is segmented into sequences of varying lengths ($N = 16, 32, 64, 128$), while keeping the total number of training samples fixed ($|X_1| \approx 3.5\text{M}$). To assess generation quality, we compute generative perplexity using GPT-2 Large and entropy on 1,024 generated samples for each NFE setting. The results are summarized in~\Tabref{tab:lm1b_gen_ppl} and~\Tabref{tab:lm1b_entropy}. For training, we follow the network hyperparameter configuration of~\citep{Schiff:2025UDLM}, modifying only the number of training iterations for each sequence dimensionality.


% \begin{table}[H]
% \small
% \centering
% \caption{Generative Perplexity ($\downarrow$) on LM1B~\citep{chelba2013exploringlm1b} measured with GPT2-large across different lengths ($D$) and pairing costs over NFE steps 4 to 1024. Best values per column are highlighted in bold.}
% \label{tab:lm1b_gen_ppl}
% \resizebox{\textwidth}{!}{
% \begin{tabular}{l|l|ccccccccc}
% \toprule
% Dimension & Method & 4 & 8 & 16 & 32 & 64 & 128 & 256 & 512 & 1024 \\
% \midrule
% \multirow{2}{*}{\shortstack[l]{$D=16$}} 
% & UDLM & 299.18 & 225.92 & 207.17 & 195.82 & 200.77 & \textbf{197.04} & 199.12 & \textbf{195.37} & 198.22 \\
% & \methodname{} & \textbf{242.22} & \textbf{208.04} & \textbf{200.99} & \textbf{190.36} & \textbf{191.74} & 199.45 & \textbf{188.84} & 196.91 & \textbf{198.12} \\
% \midrule
% \multirow{2}{*}{\shortstack[l]{$D=32$}} 
% & UDLM & 263.93 & 192.78 & 167.85 & 167.49 & 155.68 & 150.52 & 152.40 & 151.74 & 154.02 \\
% & \methodname{} & \textbf{218.48} & \textbf{172.27} & \textbf{156.35} & \textbf{150.53} & \textbf{143.83} & \textbf{145.77} & \textbf{142.54} & \textbf{141.04} & \textbf{147.57} \\
% \midrule
% \multirow{2}{*}{\shortstack[l]{$D=64$}} 
% & UDLM & 214.07 & 150.59 & 130.49 & 120.19 & 117.90 & 116.23 & 112.24 & 113.77 & 115.11 \\
% & \methodname{} & \textbf{174.78} & \textbf{138.94} & \textbf{123.06} & \textbf{115.71} & \textbf{114.73} & \textbf{112.92} & \textbf{111.29} & \textbf{107.06} & \textbf{110.83} \\
% \midrule
% \multirow{2}{*}{\shortstack[l]{$D=128$}} 
% & UDLM & 169.61 & 123.48 & 105.13 & 98.94 & 97.89 & 94.92 & 93.75 & 94.12 & 93.59 \\
% & \methodname{} & \textbf{167.90} & \textbf{121.09} & \textbf{102.16} & \textbf{96.61} & \textbf{93.93} & \textbf{91.51} & \textbf{90.21} & \textbf{89.09} & \textbf{89.07} \\
% \bottomrule
% \end{tabular}
% }
% \end{table}

\begin{table}[H]
\small
\centering
% color to blue
% \captionsetup{font={color=correctblue}} 
% \color{correctblue}
% \arrayrulecolor{correctblue}
% color to blue
\caption{Generative Perplexity ($\downarrow$) on LM1B~\citep{chelba2013lm1b} measured with GPT2-large across varying lengths ($N$) and their corresponding training iterations (Iter.) over NFE steps 4 to 1024. Best values are highlighted in bold.}
\label{tab:lm1b_gen_ppl}
\resizebox{\textwidth}{!}{
\begin{tabular}{c|c|l|ccccccccc}
\toprule
$N$ & Iter. & Method & 4 & 8 & 16 & 32 & 64 & 128 & 256 & 512 & 1024 \\
\midrule
\multirow{2}{*}{16} & \multirow{2}{*}{200k}
& UDLM & 299.18 & 225.92 & 207.17 & 195.82 & 200.77 & \textbf{197.04} & 199.12 & \textbf{195.37} & 198.22 \\
& & \methodname{} & \textbf{242.22} & \textbf{208.04} & \textbf{200.99} & \textbf{190.36} & \textbf{191.74} & 199.45 & \textbf{188.84} & 196.91 & \textbf{198.12} \\
\midrule
\multirow{2}{*}{32} & \multirow{2}{*}{200k}
& UDLM & 263.93 & 192.78 & 167.85 & 167.49 & 155.68 & 150.52 & 152.40 & 151.74 & 154.02 \\
& & \methodname{} & \textbf{218.48} & \textbf{172.27} & \textbf{156.35} & \textbf{150.53} & \textbf{143.83} & \textbf{145.77} & \textbf{142.54} & \textbf{141.04} & \textbf{147.57} \\
\midrule
\multirow{2}{*}{64} & \multirow{2}{*}{400k}
& UDLM & 214.07 & 150.59 & 130.49 & 120.19 & 117.90 & 116.23 & 112.24 & 113.77 & 115.11 \\
& & \methodname{} & \textbf{174.78} & \textbf{138.94} & \textbf{123.06} & \textbf{115.71} & \textbf{114.73} & \textbf{112.92} & \textbf{111.29} & \textbf{107.06} & \textbf{110.83} \\
\midrule
\multirow{2}{*}{128} & \multirow{2}{*}{600k}
& UDLM & 169.61 & 123.48 & 105.13 & 98.94 & 97.89 & 94.92 & 93.75 & 94.12 & 93.59 \\
& & \methodname{} & \textbf{167.90} & \textbf{121.09} & \textbf{102.16} & \textbf{96.61} & \textbf{93.93} & \textbf{91.51} & \textbf{90.21} & \textbf{89.09} & \textbf{89.07} \\
\bottomrule
\end{tabular}
}
\end{table}

\begin{table}[H]
\small
\centering
% color to blue
% \captionsetup{font={color=correctblue}} 
% \color{correctblue}
% \arrayrulecolor{correctblue}
% color to blue
\caption{Comparison of Entropy ($\uparrow$) on LM1B~\citep{chelba2013lm1b} across varying lengths ($N$) and training iterations (Iter.) over NFE steps 4 to 1024. Best values are highlighted in bold.}
\label{tab:lm1b_entropy}
\resizebox{\textwidth}{!}{
\begin{tabular}{c|c|l|ccccccccc}
\toprule
$N$ & Iter. & Method & 4 & 8 & 16 & 32 & 64 & 128 & 256 & 512 & 1024 \\
\midrule
\multirow{2}{*}{16} & \multirow{2}{*}{200k}
& UDLM & 2.46 & \textbf{2.49} & 2.50 & 2.50 & 2.50 & 2.51 & 2.50 & 2.51 & 2.50 \\
& & \methodname{} & \textbf{2.48} & \textbf{2.49} & \textbf{2.51} & \textbf{2.52} & \textbf{2.52} & \textbf{2.53} & \textbf{2.52} & \textbf{2.53} & \textbf{2.52} \\
\midrule
\multirow{2}{*}{32} & \multirow{2}{*}{200k}
& UDLM & 3.05 & 3.09 & 3.12 & 3.13 & 3.13 & 3.13 & 3.13 & 3.13 & 3.13 \\
& & \methodname{} & \textbf{3.06} & \textbf{3.12} & \textbf{3.13} & \textbf{3.14} & \textbf{3.15} & \textbf{3.15} & \textbf{3.15} & \textbf{3.15} & \textbf{3.16} \\
\midrule
\multirow{2}{*}{64} & \multirow{2}{*}{400k}
& UDLM & \textbf{3.57} & 3.63 & 3.67 & 3.68 & 3.69 & 3.70 & 3.70 & 3.69 & 3.70 \\
& & \methodname{} & \textbf{3.57} & \textbf{3.65} & \textbf{3.69} & \textbf{3.70} & \textbf{3.71} & \textbf{3.71} & \textbf{3.71} & \textbf{3.72} & \textbf{3.71} \\
\midrule
\multirow{2}{*}{128} & \multirow{2}{*}{600k}
& UDLM & 3.98 & 4.09 & 4.14 & 4.16 & 4.17 & 4.17 & 4.17 & 4.18 & 4.18 \\
& & \methodname{} & \textbf{4.00} & \textbf{4.11} & \textbf{4.16} & \textbf{4.18} & \textbf{4.19} & \textbf{4.20} & \textbf{4.20} & \textbf{4.19} & \textbf{4.20} \\
\bottomrule
\end{tabular}
}
\end{table}